Denne synopsis er en del af det indledende afklarende arbejde der
udføres før der tages beslutning om udarbejdelse af en
referencearkitektur. Formålet er at konkretisere et muligt indhold med
henblik på udpegning af interessenter samt at afgrænse opgaven i forhold
til øvrige aktiviteter. Synopsis vil, på kortest mulige form, giver et
overblik over strukturen og indholdet af den endelige arkitektur.
Synopsen er ikke et gennemtænkt bud på den endelige løsning, men skal
udtale sig om retning og afprøve rammerne for det videre arbejde.

\section{Introduktion}\label{introduktion}

\subsection{Formål}\label{formuxe5l}

Referencearkitekturen understøtter anvendelse og udviklingen af
it-systemer - der anvender og (sammenstiller) registeroplysninger til
sagsbehandling eller selvbetjening - der sender eller modtager
meddelelser fra andre it-systemer

\subsection{Scope}\label{scope}

Referencearkitekturen beskriver anvendelse af og udvikling af it-system
der reguleres af blandt andet:

\begin{description}
\tightlist
\item[EU databeskyttelse]
\emph{lov} som beskriver pligter og rettigheder
\item[EU eIDAS]
\emph{lov} som definerer registrede tillidstjenester
\end{description}

\begin{itemize}
\tightlist
\item
  DK persondata lov
\item
  Lov om Digital Post
\end{itemize}

Referencearkitekturen skrives på baggrund af - Beslutning i
Digitaliseringsstrategien - Aftale papir og deltagerkreds

Uden for scope: - registrering og anvendelse hos registerejer - kun
registre og dokumenter der kræver adgangskontrol

\subsection{Centrale begreber}\label{centrale-begreber}

Register, registerejer, dataansvarlig, dataanvender, den registrerede,
Grunddata, Dokument, Afsender, modtager, meddelelse

\subsection{Anvendelse}\label{anvendelse}

\begin{itemize}
\tightlist
\item
  bruges sprog til at formulere en fælles strategi
\item
  bruges som reference ved løsningsbeskrivelser
\end{itemize}

\subsection{Tilblivelse og governance}\label{tilblivelse-og-governance}

Denne version er skrevet\ldots{}.og rettet mod ``dem der laver
strategi'' og it-arkitekter Endelig godkendelse hos SDA

\subsection{Metoderamme}\label{metoderamme}

Skrives indefor rammerne af Fællesoffentlige Digital Arkitektur, det vil
sige\ldots{} - erfaringer fra OIO referencearkitektur - og EIRA -
TOGAFF, ArchiMate

\subsection{Relation til andre
referencearkitekturer}\label{relation-til-andre-referencearkitekturer}

Gør brug af - Fællesoffentlig referencearkitektur for brugerstyring

Skal kunne anvendes af - Fællesoffentlig referencearkitektur for
selvbetjening - Fællesoffentlig referencearkitektur for overblik over
egne sager

Skal anvendes i kontekst sammen med - Deling af dokumenter på
sundhedsområdet - Indberetning til registre på sundhedsområdet - Sag- og
dokument på det kommunale område

\section{Strategi? (Vision, mål og
strategier)}\label{strategi-vision-muxe5l-og-strategier}

\subsection{Forretningsmæssige
tendenser}\label{forretningsmuxe6ssige-tendenser}

\begin{itemize}
\tightlist
\item
  Ensretning og nationale indsatser
\item
  Data øget værdi for organisationer
\item
  Øget bevågenhed omkring beskyttelse af privatliv
\item
  Øget opmærksomhed om håndtering af personlige oplysninger
\item
  Mængden af oplysninger der håndteres stiger
\item
  Grænseoverskridende services
\end{itemize}

\subsection{Teknologiske tendenser}\label{teknologiske-tendenser}

\begin{itemize}
\tightlist
\item
  øget central standardisering af begreber, datamodeller og grænseflader
\item
  Flere og mere forskelligartede enheder forbundet til netværket
\item
  Øgede forventninger til brugervenlighed af offentlige digitale
  services
\item
  Mængden af tilgængelige oplysninger vokser
\item
  Arkitekturvision for anvendelse og udstilling
\item
  Intergrated Service Delivery
\item
  ''Once only''
\item
  ''Ineroperability/Samarbejdende infrastrukturer / Økosystem af fælles
  løsninger?''
\item
  ''Valgfri for anvender mellem flere tekniske udbydere af samme
  oplysninger''
\end{itemize}

\subsection{Målsætning}\label{muxe5lsuxe6tning}

{[}beskriv målsætninger i eksisterende aftaler og strategier{]}

\begin{description}
\tightlist
\item[Interoperability]
\emph{mål} om sammenhængende services\ldots{} integrated service
delivery
\item[Once-only]
\emph{mål} om at borger og virksomhed kun skal afgive den samme
information til det offentlige en gang\ldots{} (men give lov til
genbrug?)
\item[Transperancy]
\emph{mål} om borger og virksomheder skal kunne se hvilke data der
findes om dem og hvor disse data anvendes
\item[Re-use]
\emph{mål} om genbrug af it med henblik på lavere omkostninger
\end{description}

\subsection{Vision}\label{vision}

\begin{verbatim}
data deles på en måde hvor dataejer ikke unødigt begrænser genbrug...
(prøve at ramme høste-så problemet og sikre gennemsigtighed og beskyttelse )
\end{verbatim}

\subsection{Værdiskabelse}\label{vuxe6rdiskabelse}

Mindre besvær for borger og virksomheder ved brug af digitale services
Simplere arbejdsgange og mere potentiale for automatisering hos
myndigheder {[}og virksomheder{]} Understøtte transparens og bevare
tillid til registre Effektiv systemudvikling (begrænse udfaldsrum,
opsamle best practice)

\subsection{Strategiske principper}\label{strategiske-principper}

\begin{itemize}
\tightlist
\item
  F1: Autoritative register med henvisninger til andre registre
\item
  F2: Ansvar for begrænsning af adgang ligger hos registerejer
\item
  F3: Let at komme med forslag til rettelser
\item
  I1: Fælles referenceinformationsmodel
\item
  I2: Dokument-princip (attester mv.)?
\item
  A1: Onlineopslag i sagsbehandling og selvbetjening
\item
  A2: Log adgang
\item
  A3: Adgang til og fra internationale registre sker gennem national
  gateway
\item
  T1: Central fuldmagt/rettighedsstyring
\item
  T2: Multi-flavour-api
\end{itemize}

\section{Forretning}\label{forretning}

\subsection{Opgaver}\label{opgaver}

\begin{itemize}
\tightlist
\item
  Borger og virksomhedsvendte selvbetjeningsløsninger
\item
  Myndigheders sagsbehandling
\item
  Tværgående analyse, tilsyn, kontrol
\end{itemize}

\subsection{Funktioner}\label{funktioner}

Referencearkitketuren beskriver tre centrale use cases hvor aktører
arbejder sammen\ldots{}

Registrering \textasciitilde{} \emph{funktion} hvor oplysninger bringes
på digital form

Datanvendelse \textasciitilde{} \emph{funktion} hvor oplysninger
anvendes i en opgave

Registreret forsendelse \textasciitilde{} \emph{funktion} hvor
meddelelser sendes uafviseligt

\subsection{Aktører, roller}\label{aktuxf8rer-roller}

\begin{itemize}
\tightlist
\item
  Borger, virksomhed, offentlig myndigheder
\item
  Udlandske?
\end{itemize}

Registrant \textasciitilde{} \emph{rolle} som bringer oplysninger på
digital form, registrer

\begin{description}
\tightlist
\item[Dataejer]
\emph{rolle} som ejer registreringer/data
\item[Dataanvender]
\emph{rolle} der anvender oplysninger fra et register
\item[Datasubject]
\emph{rolle} som oplysninger handler
\end{description}

\subsection{Tværgående processer (proces-trin, business
functions?)}\label{tvuxe6rguxe5ende-processer-proces-trin-business-functions}

Herunder beskrives relevante proces-trin fra processer beskrevet andre
steder.

\begin{itemize}
\item
  Sagsbehandling (fra sag og dokument):
\item
  Simpel selvbetjening (fra selvbetjening):
\item
  Tværgående selvbetjening (fra sammenhængende services):
\item
  Indsigt i oplysninger og deres anvendelse (fra overblik?)
\item
  Sende meddelelse
\item
  Modtage meddelelse
\item
  Tag et dokument med til en anden service provider (der ikke har adgang
  til registre)
\end{itemize}

\subsection{Forretnings-tjenester?
-funktioner?}\label{forretnings-tjenester--funktioner}

Proces trin kan realiseres af interne busines functions eller trække på
eksterne business services. Skal vi bare slå swervices og functions
sammen (da vi ikke taler om implementering endnu) - Dataudstilling -
Forsendelse - Brugerstyring

Nødvendige: Dataservice(Register), eDelivery Service, Katalog,
Kontaktregister,, Log(Overblik). Ønskelige: Signering, Distributør,
Indeks Mangler: Referencedata (Klassifikation), Identitet/brugerstyring

\begin{description}
\tightlist
\item[Dataindeksejer]
\emph{rolle} som er ansvarlig for opbevaring af metadata
\item[Datadistributør]
\emph{rolle} som er ansvarlig for adgang til data for dataanvendere
\end{description}

\subsection{Forretningsobjekter}\label{forretningsobjekter}

\subsubsection{Data}\label{data}

Abstrakt\ldots{}bruges om både registerrecord og dokument

\subsubsection{Registeroplysning
(record)}\label{registeroplysning-record}

\subsubsection{Dokument}\label{dokument}

\subsubsection{Datasamling}\label{datasamling}

\subsubsection{Datasubjekt}\label{datasubjekt}

\subsubsection{Indeks}\label{indeks}

\subsubsection{Katalog}\label{katalog}

\subsubsection{Model/Schema}\label{modelschema}

\subsubsection{Segl}\label{segl}

mangler

Meddelelse, Påmindelse, Registreringshændelse

\section{Teknik}\label{teknik}

forretningsfunktionerne understøttes/realiseres af applikationer.

\subsection{Applikationsroller}\label{applikationsroller}

\subsubsection{eDelivery Service
Provider}\label{edelivery-service-provider}

som skal kunne: - udstille eller levere meddelelser til modtager -
modtage og distribuere meddeleleser - fortælle andre om deres kunder

\subsubsection{Dataservice}\label{dataservice}

som skal kunne: - opbevare datasamling - begrænse adgang til de rigtige
- måske vedligeholde og udsende abonnementer

\subsubsection{Kontaktregister}\label{kontaktregister}

som er en slags data service med en særlig type oplysninger

\subsubsection{Log}\label{log}

som er en slags data service med særlige oplysninger

\subsubsection{Indeks}\label{indeks-1}

som er en slags data service med særlige oplysninger kan undværes, men
ikke effektivt.

\subsection{Katalog}\label{katalog-1}

som ikke er en dataservice fordi der ikke er begrænset adgang kan
undværes, men ikke effektivt.

(Skal vi have en ``beskyttet dataservice'' og en offentlig?)

\section{Implementering(er)}\label{implementeringer}

Her placeres de enkelte services på processtrin fra tidligere afsnit

\subsection{Datanvendelse}\label{datanvendelse}

Når myndighed vil have adgang til data hos en anden er det er par
mønstre

\subsubsection{Direkte adgang, SOA}\label{direkte-adgang-soa}

\subsubsection{Datadistribution}\label{datadistribution}

sammenstilling samt adgangskontrol og logning

\subsubsection{Distribueret Service- og
data-platform}\label{distribueret-service--og-data-platform}

\subsection{Registreret forsendelse}\label{registreret-forsendelse}

Når en myndighed vil sende noget til en myndighed, virksom eller borger.

\subsubsection{SOA / Email\ldots{}}\label{soa-email}

\subsubsection{Fælles system}\label{fuxe6lles-system}

e.g.~e-Boks.

\subsubsection{Service Providers}\label{service-providers}

kan være både generisk eller specifik for et domæne.

\subsection{Registrering}\label{registrering}

skal med for at forklare index

\subsection{ansvar hos registrant}\label{ansvar-hos-registrant}

\subsection{ansvar hos dataejer}\label{ansvar-hos-dataejer}

\subsection{Områder for
standardisering/profileringer}\label{omruxe5der-for-standardiseringprofileringer}

(Per mønster?, matrix) - Service Design Guidelines - Access Protocols -
Distribution Protocols - Synchronisation Protocols

\begin{itemize}
\tightlist
\item
  Metadata for opslag/søgning/anvendelse
\item
  Log format
\item
  Identifikation
\item
  Klassifikation af følsomhed
\item
  Klassifikation af anvendelse (sagsbehandling vs analyse)
\item
  Hændelsesbeskeder
\item
  Protokol for flytning af filer, kryptering
\item
  Hjemmel (samtykke, lov)
\item
  Context
\end{itemize}

\subsection{Identifikation af
standarder}\label{identifikation-af-standarder}
